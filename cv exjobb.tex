\documentclass[10pt]{article}
\usepackage{array, xcolor, lipsum, bibentry}
\usepackage[a4paper, top=2cm, bottom=2cm]{geometry}


\usepackage{multirow}


\usepackage[T1]{fontenc}	      
\usepackage[swedish]{babel}
\usepackage[utf8]{inputenc}

\usepackage{libertine}
\usepackage{hyperref}
\hypersetup{
colorlinks=true
}

\title{Kristofer Leifland}
\author{}
\date{}
\definecolor{lightgray}{gray}{0.8}
\newcolumntype{L}{>{\raggedleft}p{0.14\textwidth}}
\newcolumntype{R}{p{0.8\textwidth}}
\newcommand\VRule{\color{lightgray}\vrule width 0.5pt}
\begin{document}
\maketitle
\thispagestyle{empty}
\pagestyle{empty}
\noindent
Kristofer Leifland <\href{mailto:kristofer.leifland@gmail.com}{\nolinkurl{kristofer.leifland@gmail.com}}>\\
Ystadsgatan 23C Lgh 1301, 214 24 Malmö\\
0708 184795\\
%\begin{minipage}[ht]{0.48\textwidth}
%January 3rd, 2020\\
%+12 34 56 789
%\end{minipage}

\section*{Mål}
Jag är en blivande dataingenjör med stort intresse för programmering som söker examensjobb inför våren 2014. Jag började programmera på högstadiet och har gjort projekt inom bland annat C++, Java, C\#, R, MatLab och Scala. På mitt GitHub-konto finns några exempel på kod som jag har skrivit, se \url{http://github.com/peoplesbeer}. Jag har sedan tidigare en magisterexamen i Miljövetenskap, men bestämde mig efter att ha arbetat ett par år att fördjupa mitt intresse och sadla om till ingenjör.

\section*{Yrkeserfarenhet}
\begin{tabular}{L!{\VRule}R}
2012--Nu&{\bf Labhandledning i datorvetenskap.}\\
&Handledning av studenter under laborationer samt bedömning av inlämningsuppgifter inom kurser i datavetenskap.\\
Hösten 2012&{\bf Support function på Syngenta Seeds AB (sockerbetsförädling).}\\
&Diverse uppgifter kopplade till ett mjukvarumigreringsprojekt.

\noindent \begin{tabular}{l}
- Ansvarig för att utforma en frökvalitetsrapport. \\
- Migrering av en Accessdatabas och Excel VBA-kod för att passa nya format. \\
- Utveckling av rapporter i Crystal Reports. \\
\end{tabular}
\\
Sommar 2012&{\bf Praktik som webutvecklare på 23 Gears i Göteborg.}\\
&Jag utvecklade en webapplikation i ASP.NET MVC 3 som gör att användare kan skapa och söka ärenden i ärendehanteringsystemet Jira (genom Jiras REST API). \\
2010--2012&{\bf Product evaluation assistant på Syngenta Seeds AB (sockerbetsförädling).}\\
&På Syngenta fick jag möjligheten att arbeta i några spännande projekt med deras statistiker (Rikard Alm) som omfattade data mining, statistisk analys och visualisering av stora dataset. Jag var också involverad i ett stort mjukvaruprojekt där jag samarbetade med utvecklare i USA och Egypten och bidrog med testning, utveckling av rapporter i Crystal Reports samt användardokumentation. Mina rutinuppgifter var att hantera data och assistera i försöksplaneringen med produktutvecklingsgruppen.\\
\end{tabular}

\section*{Utbildning}
\begin{tabular}{L!{\VRule}R}
2011--2014 & {\bf Högskoleingenjör i data och telekommunikation, Malmö Högskola.}\\
&Det har varit skoj att få chansen att förkovra sig i ett ämne som man brinner för. Jag är främst intresserad av programmering och tagit varje chans att bli en bättre programmerare. \\
2002--2008&{\bf Magisterprogram i Miljövetenskap, Lunds universitet.}\\
&I mitt examensarbete utvärderade jag flera biomassmodeller för kräftfiske med hjälp av egenutvecklade verktyg för statistisk analys i MatLab. Exjobbet finns att läsa här, \url{http://goo.gl/DtSNY}. \\
%\vspace{5pt}\\
\end{tabular}

%\section*{Languages}
%\begin{tabular}{L!{\VRule}R}
%Swedish&Mother tongue\\
%English&Fluent\\
%French&Some\\
%\end{tabular}

\end{document}
